\documentclass[a4paper,12pt]{article}
\usepackage[utf8]{inputenc}
\usepackage[russian]{babel}
\usepackage{geometry}
\usepackage{minted}
\usepackage{graphicx}
\geometry{top=2cm,bottom=2cm,left=2.5cm,right=2.5cm}
\begin{document}

\begin{center}
\textbf{Федеральное государственное автономное образовательное учреждение высшего образования}\\
\textbf{<<НАЦИОНАЛЬНЫЙ ИССЛЕДОВАТЕЛЬСКИЙ 
УНИВЕРСИТЕТ ИТМО
>>}\\[0.5em]
Факультет программной инженерии и компьютерной техники (ПИиКТ)\\
\end{center}

\vspace{7cm}

\begin{center}
\textbf{Теория функций комплексных переменных}\\[0.5em]
Лабораторная работа № 1\\[0.5em]
\end{center}

\vfill

\hfill
\begin{minipage}{0.95\textwidth}
Выполнили\\
Макаров Матвей\\
Красногорский Тимофей\\[1em]
Преподаватель: Краснов Александр Юрьевич
\end{minipage}

\vspace{1cm}

\begin{center}
г. Санкт-Петербург\\
2025 г. 
\end{center}
\pagebreak

\begin{center}
    \textbf{Задание 1}
\end{center}

Доказательство свойства 1

Доказать: Множество переходит само в себя при сопряжении. \newline

Рассмотрим последовательности: \newline
         $z_{n+1} = z_n² + c, z₀ = 0$ \newline
         $w_{n+1} = w_n² + c̄, w₀ = 0$ \newline
         
По математической индукции: \newline
1. База. n = 0 \newline
 $z_0 = 0$ \newline
\ $w_0 = 0 = \overline{0}=\overline{z_0}$ \newline
2. Предположение. n = k \newline
 \ $ w_n = \overline{z_n}$ \newline
3. Шаг. $n = k+1$ \newline
\ $w_{k+1}=\overline{z_{k+1}}=\overline{z_{n}^2+c}=\overline{z_n}^2+\overline{c}=w_n^2+\overline{c}$ \newline

Тогда для всех n выполняется $|w_n|=|z_n|$. Значит, если последовательность $z_n$ ограничена при $c \in M$ (по определению), то $w_n$ тоже ограничена, откуда $\overline{c} \in M$. Тогда множество Мандельброта симметрично относительно вещественной оси. \newline

Доказательство свойства 2 \newline
Доказать, что $\forall c \in C: |c|>2 \leftrightarrow  c \notin M$ \newline
Определение ограниченность $z_n$: \newline
$\exists R > 0 : \forall n \in N, |z_n| \leq R$ \newline
Возьмем произвольный c : c > 2. \newline
$|z_{n+1}|=|z_n^2+c|\geq|z_n|^2-|c|$ \newline

По математической индукции: \newline
1. База. Пусть n = 1, тогда $|z_1|=|c|, |c|>2$. \newline
$|z_2|\geq|z_1|^2-|c|$ \newline
$|z_2|\geq |c|^2 -|c|$ \newline
$[|c|-1 >1]$ \newline
$|z_2|>|c|$ \newline
$|z_2|>2$ \newline
2. Предположение. При n = k \newline
$|z_{k+1}|>|z_k|$ \newline
3. Шаг.  \newline
$|z_{k+1}|\geq |z_k|^2-|z_k|$ \newline
$|z_k|-1>1$ \newline
$|z_{k+1}|>|z_k|$ \newline

Значит, при любом n выполняется $|z_{n+1}|>|z_n|$, тогда при $n\rightarrow \infty$ и |c|>2 $z_n$ не ограничена.    \newline

\newpage

\begin{center}
    \textbf{Задание 2}
\end{center}
Программа для изображения множества Мандельброта.

\begin{minted}[frame=single,fontsize=\small,linenos]{python}
import numpy as np
import matplotlib.pyplot as plt

def mandelbrot_set(xmin, xmax, ymin, ymax, width, height, max_iter):
    x = np.linspace(xmin, xmax, width)
    y = np.linspace(ymin, ymax, height)
    X, Y = np.meshgrid(x, y)
    C = X + Y * 1j

    mandelbrot = np.zeros(C.shape, dtype=int)
    Z = np.zeros(C.shape, dtype=complex)

    for i in range(max_iter):
        mask = np.abs(Z) <= 2
        Z[mask] = Z[mask] * Z[mask] + C[mask]
        mandelbrot[mask] = i

    mandelbrot[np.abs(Z) <= 2] = max_iter
    return mandelbrot

def plot_mandelbrot(xmin, xmax, ymin, ymax, width=800, height=800, max_iter=100):
    mandelbrot = mandelbrot_set(xmin, xmax, ymin, ymax, width, height, max_iter)

    plt.figure(figsize=(10, 8))
    plt.imshow(mandelbrot, extent=[xmin, xmax, ymin, ymax], cmap='magma', origin='lower')
    plt.colorbar(label='Итерации')
    plt.title('Множество Мандельброта')
    plt.tight_layout()
    plt.show()

if __name__ == "__main__":
    xmin, xmax = -0.748, -0.747
    ymin, ymax = 0.102, 0.103
    max_iter = 2000
    plot_mandelbrot(xmin, xmax, ymin, ymax, max_iter=max_iter)
\end{minted}

\newpage

\begin{center}
    \textbf{Результаты для множества Мандельброта}
\end{center}

1. Стандартный вид

\begin{figure}[h!]
    \centering
    \includegraphics[width=0.8\linewidth]{photo_2025-10-08_01-45-24.jpg}
    \caption{Множество Мандельброта: стандартный вид \\ xmin, xmax = -2.0, 1.0 \\ ymin, ymax = -1.5, 1.5}
    \label{fig:mandel1}
\end{figure}

\newpage

2. Спирали

\begin{figure}[h!]
    \centering
    \includegraphics[width=0.8\linewidth]{photo_2025-10-08_01-44-04.jpg}
    \caption{Множество Мандельброта: спирали \\ xmin, xmax = 0.275, 0.285 \\ ymin, ymax = 0.005, 0.015}
    \label{fig:mandel2}
\end{figure}

\newpage

3. Морской конек

\begin{figure}[h!]
    \centering
    \includegraphics[width=0.8\linewidth]{photo_2025-10-08_01-50-27.jpg}
    \caption{Множество Мандельброта: морской конек \\ xmin, xmax = -0.748, -0.747 \\ ymin, ymax = 0.102, 0.103}
    \label{fig:mandel3}
\end{figure}

\newpage

\begin{center}
    \textbf{Задание 3}
\end{center}

\begin{minted}[frame=single,fontsize=\small,linenos]{python}
import numpy as np
import matplotlib.pyplot as plt

def julia_set(c, xmin, xmax, ymin, ymax, width, height, max_iter):
    x = np.linspace(xmin, xmax, width)
    y = np.linspace(ymin, ymax, height)
    X, Y = np.meshgrid(x, y)
    Z = X + Y * 1j

    julia = np.zeros(Z.shape, dtype=int)

    for i in range(max_iter):
        mask = np.abs(Z) <= 2
        Z[mask] = Z[mask] * Z[mask] + c
        julia[mask] = i

    julia[np.abs(Z) <= 2] = max_iter
    return julia

def plot_julia(c, xmin, xmax, ymin, ymax, width=800, height=800, max_iter=100):
    julia = julia_set(c, xmin, xmax, ymin, ymax, width, height, max_iter)

    plt.figure(figsize=(10, 8))
    plt.imshow(julia, extent=[xmin, xmax, ymin, ymax], cmap='magma', origin='lower')
    plt.colorbar(label='Итерации')
    plt.title(f'Множество Жюлиа для c = {c}')
    plt.tight_layout()
    plt.show()

if __name__ == "__main__":
    c = -0.8j
    xmin, xmax = -1.5, 1.5
    ymin, ymax = -1.5, 1.5
    max_iter = 300
    plot_julia(c, xmin, xmax, ymin, ymax, max_iter=max_iter)
\end{minted}

\newpage

\begin{center}
    \textbf{Результаты для множества Жюлиа}
\end{center}

1. Множество Жюлиа для c = -0.7 + 0.27015j

\begin{figure}[h!]
    \centering
    \includegraphics[width=0.8\linewidth]{photo_2025-10-08_01-58-15.jpg}
    \caption{Множество Жюлиа для c = -0.7 + 0.27015j \\ xmin, xmax = -1.5, 1.5 \\ ymin, ymax = -1.5, 1.5}
    \label{fig:julia1}
\end{figure}

\newpage

2. Множество Жюлиа для c = -0.4 + 0.6j

\begin{figure}[h!]
    \centering
    \includegraphics[width=0.8\linewidth]{photo_2025-10-08_01-58-49.jpg}
    \caption{Множество Жюлиа для c = -0.4 + 0.6j \\ xmin, xmax = -1.5, 1.5 \\ ymin, ymax = -1.5, 1.5}
    \label{fig:julia2}
\end{figure}

\newpage

3. Множество Жюлиа для c = -0.8j

\begin{figure}[h!]
    \centering
    \includegraphics[width=0.8\linewidth]{photo_2025-10-08_02-00-36.jpg}
    \caption{Множество Жюлиа для c = -0.8j \\ xmin, xmax = -1.5, 1.5 \\ ymin, ymax = -1.5, 1.5}
    \label{fig:julia3}
\end{figure}

\end{document}